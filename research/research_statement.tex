\documentclass[12pt]{article}
\usepackage[whole]{bxcjkjatype}
\usepackage{epsf}
\usepackage{amsmath,amssymb}
\usepackage{bm}

\usepackage{graphicx}

\usepackage{comment}
\usepackage{multirow}
\usepackage{braket}

\usepackage{hyperref}
\usepackage{tgtermes}
\usepackage{float}
\usepackage{xcolor}
\usepackage{siunitx}
\usepackage{mhchem}
\usepackage{cite}

\usepackage{cleveref}

% set margin
\usepackage[margin=1in]{geometry}

% remove the page number
% \pagestyle{empty}

% set section title all caps
\usepackage{titlesec}
\titleformat*{\section}{\large\bf}
\titleformat*{\subsection}{\normalsize\bf}
\titlespacing*{\section}{0pt}{1.6em}{1em}
\titlespacing*{\subsection}{0pt}{1.6em}{1em}

% publications title
\renewcommand{\refname}{References}

% base line stretch
\renewcommand{\baselinestretch}{1.1}
\allowdisplaybreaks[1]


% customize the title
\usepackage{titling}
\pretitle{\vspace{-1in}}
\title{{\Large Research Statement}}
\posttitle{
  \par
}
\preauthor{\vspace{0.5em}}
\author{
  \indent{\large So Chigusa}\\
  \indent\textit{schigusa@mit.edu}\\
  \indent\textit{Massachusetts Institute of Technology, 77 Massachusetts Avenue, Cambridge, MA 02139}
}
\postauthor{}
\date{\vspace{-3em}}


% comments block
\def\rem#1{ {\bf\textcolor{red}{($\clubsuit$ #1 $\clubsuit$)}}}


\begin{document}
\maketitle

My research lies at the intersection of quantum science and high-energy physics.
Recent advances in quantum technologies are transforming how we approach fundamental physics questions.
Among them, \textbf{quantum sensing} offers powerful methods for detecting faint signals, while \textbf{quantum computation} enables the simulation of complex dynamics by directly manipulating quantum states.
Building on these developments, my work focuses on two complementary directions:
\textit{(i) new physics searches with quantum sensing}, and \textit{(ii) quantum simulation of parton shower dynamics}.

\subsection*{New physics searches with quantum sensing}

One of my research areas focuses on developing methods to search for \textbf{light dark matter} using quantum sensing techniques.
Conventional direct detection experiments, which primarily target the $\mathrm{GeV}$ mass range, have not yet provided evidence for dark matter.
This has motivated the community to explore a broader parameter space, including the sub-$\mathrm{GeV}$ regime, which remains largely unexplored due to the challenges of low excitation energies and small event rates.
Quantum sensing offers a promising path toward detecting such faint signals.
By leveraging these techniques, I aim to overcome current limitations in sensitivity and frequency coverage, opening new opportunities for the discovery of light dark matter.

\begin{figure}[t]
  \centering
  \includegraphics[width=0.6\hsize]{../public/rs/Summary_Plot_2025.png}
  \caption{
    Summary of the frequency coverage of various approaches discussed in the main text.
    The prospects for axion dark matter are shown for illustration.
    Each result, represented by a solid or dashed line, can be compared with the current constraint, plotted as a dotted line of the same color for the corresponding coupling.
  }
  \label{fig:summary}
\end{figure}

My research explores multiple collective spin excitations, \textbf{magnon} \cite{Chigusa:2020gfs}, \textbf{axion} \cite{Chigusa:2021mci}, and \textbf{nuclear magnon} \cite{Chigusa:2023hmz}, to probe diverse dark matter couplings, as illustrated by the solid lines in \cref{fig:summary}.
These approaches provide valuable sensitivity in the sub-$\mathrm{THz}$ regime.
In parallel, I proposed new searches using \textbf{nitrogen-vacancy center} magnetometry \cite{Chigusa:2023roq}, \cite{Chigusa:2024psk}, which offers broad frequency coverage and sensitivities to different spin channels (dashed lines in \cref{fig:summary}).
I teamed up with experimental experts, and together we have recently demonstrated data-analysis techniques for incoherent signals \cite{Herbschleb:2024pbk}.
Our experiment is now advancing toward cryogenic operation, and we expect first results by the end of 2026.

These quantum sensing approaches reach their full potential by harnessing non-classical resources of quantum states.
I have investigated methods to surpass the standard quantum limit in dark matter detection using \textbf{squeezing} \cite{Chigusa:2023szl} and \textbf{entanglement} \cite{Sichanugrist:2024wfk}.
Notably, for frequency-scan searches targeting signals with unknown frequencies, I found that entangled states can enhance sensitivity even in the presence of Markovian noise \cite{Sichanugrist:2024wfk}.
Beyond these, I am developing \textbf{sensing protocols} tailored to dark matter searches by integrating quantum-state control with measurement.
My recent work based on nitrogen-vacancy center \cite{Chigusa:2025ihd} proposes a protocol that suppresses magnetic noise as a decoupling protocol while maintaining broadband sensitivity.
More broadly, pursuing quantum sensing protocols explicitly designed around the distinctive features of dark matter signals represents a promising research direction for the next generation experiements.

The relevance of quantum sensing extends beyond dark matter.
Relativistic targets such as high-frequency gravitational waves or cosmic axion background, as well as searches for a fifth force, offer additional directions.
By the nature of fundamental physics researches, different new-physics scenarios favor distinct detection systems and sensing protocols.
Moreover, quantum sensing can not only detect signals but also reveal their underlying nature.
Examples include \textbf{harvesting quantumness} of wave-like dark-matter signals and mitigating \textbf{look-elsewhere effects} through correlations between neighboring qubits, both achievable through appropriate sensing protocols.
Overall, quantum sensors hold remarkable potential as versatile tools for addressing a wide range of challenges in new physics searches.

\subsection*{Quantum simulation of parton shower dynamics}

\begin{figure}[t]
  \centering
  \includegraphics[width=0.5\hsize]{../public/rs/shower_annotated.png}
  \caption{
    A schematic illustration of multi-emission processes in parton shower simulations.
    Blue cones represent independent collinear emissions included in the current algorithms, while orange lines indicate soft radiation that generates global event-wise entanglement.
  }
  \label{fig:shower}
\end{figure}

Another direction of my research focuses on developing quantum algorithms to study the dynamics of quantum fields and particles.
Quantum computing resources with a substantial number of qubits are now publicly accessible and steadily improving in quality and scalability.
This rapid progress makes it an ideal time to explore how quantum algorithms can advance physics research.
My work leverages these developments to push the boundaries of quantum simulation, aiming to capture complex quantum processes that are beyond the reach of classical computation.

As a representative system exhibiting rich quantum behavior, I study \textbf{parton showers}.
Conventional parton shower algorithms, classical tools long used to simulate multi-emission processes in collider and astroparticle physics (see \cref{fig:shower}), break down when \textbf{quantum interference effects} become significant, especially in systems with nontrivial flavor structure \cite{Chigusa:2022act}.
To overcome this, I developed a \textbf{quantum parton shower algorithm} based on veto procedures \cite{Bauer:2023ujy}, capable of incorporating exponentially many interference diagrams using only polynomial quantum resources.

The quantum simulation algorithms developed in these works represent an initial step toward realistic quantum parton shower simulations.
Many challenges remain before such simulations can be applied to phenomenological studies, including the incorporation of \textbf{soft and color interference} effects.
Embedding these effects within my quantum simulation framework requires encoding additional degrees of freedom, such as spin, color, and emission history, into qubits.
These extensions are not only algorithmically significant but also physically essential, as quantum interference underlies key observables of fundamental importance across both particle and nuclear physics, including \textbf{electroweak showers} and \textbf{energy correlators}.
Ultimately, this line of research aims to establish a comprehensive quantum framework capable of describing realistic collider events, bridging quantum computation and quantum field theory dynamics.

\subsection*{Conclusion}

The program for exploring new physics must evolve in step with the rapid technological progress of quantum science.
By integrating advanced quantum sensing techniques and developing quantum algorithms, I aim to create innovative methods for investigating fundamental phenomena and to contribute to a deeper understanding of the universe's underlying principles.
In the long term, my goal is to establish a research framework where quantum technology and high-energy physics advance together, driving discovery across both theory and experiment.

\bibliographystyle{JHEP}
\bibliography{publications.bib}

\end{document}
